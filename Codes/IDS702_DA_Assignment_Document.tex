% Options for packages loaded elsewhere
\PassOptionsToPackage{unicode}{hyperref}
\PassOptionsToPackage{hyphens}{url}
\PassOptionsToPackage{dvipsnames,svgnames,x11names}{xcolor}
%
\documentclass[
  letterpaper,
  DIV=11,
  numbers=noendperiod]{scrartcl}

\usepackage{amsmath,amssymb}
\usepackage{iftex}
\ifPDFTeX
  \usepackage[T1]{fontenc}
  \usepackage[utf8]{inputenc}
  \usepackage{textcomp} % provide euro and other symbols
\else % if luatex or xetex
  \usepackage{unicode-math}
  \defaultfontfeatures{Scale=MatchLowercase}
  \defaultfontfeatures[\rmfamily]{Ligatures=TeX,Scale=1}
\fi
\usepackage{lmodern}
\ifPDFTeX\else  
    % xetex/luatex font selection
\fi
% Use upquote if available, for straight quotes in verbatim environments
\IfFileExists{upquote.sty}{\usepackage{upquote}}{}
\IfFileExists{microtype.sty}{% use microtype if available
  \usepackage[]{microtype}
  \UseMicrotypeSet[protrusion]{basicmath} % disable protrusion for tt fonts
}{}
\makeatletter
\@ifundefined{KOMAClassName}{% if non-KOMA class
  \IfFileExists{parskip.sty}{%
    \usepackage{parskip}
  }{% else
    \setlength{\parindent}{0pt}
    \setlength{\parskip}{6pt plus 2pt minus 1pt}}
}{% if KOMA class
  \KOMAoptions{parskip=half}}
\makeatother
\usepackage{xcolor}
\setlength{\emergencystretch}{3em} % prevent overfull lines
\setcounter{secnumdepth}{-\maxdimen} % remove section numbering
% Make \paragraph and \subparagraph free-standing
\ifx\paragraph\undefined\else
  \let\oldparagraph\paragraph
  \renewcommand{\paragraph}[1]{\oldparagraph{#1}\mbox{}}
\fi
\ifx\subparagraph\undefined\else
  \let\oldsubparagraph\subparagraph
  \renewcommand{\subparagraph}[1]{\oldsubparagraph{#1}\mbox{}}
\fi


\providecommand{\tightlist}{%
  \setlength{\itemsep}{0pt}\setlength{\parskip}{0pt}}\usepackage{longtable,booktabs,array}
\usepackage{calc} % for calculating minipage widths
% Correct order of tables after \paragraph or \subparagraph
\usepackage{etoolbox}
\makeatletter
\patchcmd\longtable{\par}{\if@noskipsec\mbox{}\fi\par}{}{}
\makeatother
% Allow footnotes in longtable head/foot
\IfFileExists{footnotehyper.sty}{\usepackage{footnotehyper}}{\usepackage{footnote}}
\makesavenoteenv{longtable}
\usepackage{graphicx}
\makeatletter
\def\maxwidth{\ifdim\Gin@nat@width>\linewidth\linewidth\else\Gin@nat@width\fi}
\def\maxheight{\ifdim\Gin@nat@height>\textheight\textheight\else\Gin@nat@height\fi}
\makeatother
% Scale images if necessary, so that they will not overflow the page
% margins by default, and it is still possible to overwrite the defaults
% using explicit options in \includegraphics[width, height, ...]{}
\setkeys{Gin}{width=\maxwidth,height=\maxheight,keepaspectratio}
% Set default figure placement to htbp
\makeatletter
\def\fps@figure{htbp}
\makeatother

\KOMAoption{captions}{tableheading}
\makeatletter
\makeatother
\makeatletter
\makeatother
\makeatletter
\@ifpackageloaded{caption}{}{\usepackage{caption}}
\AtBeginDocument{%
\ifdefined\contentsname
  \renewcommand*\contentsname{Table of contents}
\else
  \newcommand\contentsname{Table of contents}
\fi
\ifdefined\listfigurename
  \renewcommand*\listfigurename{List of Figures}
\else
  \newcommand\listfigurename{List of Figures}
\fi
\ifdefined\listtablename
  \renewcommand*\listtablename{List of Tables}
\else
  \newcommand\listtablename{List of Tables}
\fi
\ifdefined\figurename
  \renewcommand*\figurename{Figure}
\else
  \newcommand\figurename{Figure}
\fi
\ifdefined\tablename
  \renewcommand*\tablename{Table}
\else
  \newcommand\tablename{Table}
\fi
}
\@ifpackageloaded{float}{}{\usepackage{float}}
\floatstyle{ruled}
\@ifundefined{c@chapter}{\newfloat{codelisting}{h}{lop}}{\newfloat{codelisting}{h}{lop}[chapter]}
\floatname{codelisting}{Listing}
\newcommand*\listoflistings{\listof{codelisting}{List of Listings}}
\makeatother
\makeatletter
\@ifpackageloaded{caption}{}{\usepackage{caption}}
\@ifpackageloaded{subcaption}{}{\usepackage{subcaption}}
\makeatother
\makeatletter
\@ifpackageloaded{tcolorbox}{}{\usepackage[skins,breakable]{tcolorbox}}
\makeatother
\makeatletter
\@ifundefined{shadecolor}{\definecolor{shadecolor}{rgb}{.97, .97, .97}}
\makeatother
\makeatletter
\makeatother
\makeatletter
\makeatother
\ifLuaTeX
  \usepackage{selnolig}  % disable illegal ligatures
\fi
\IfFileExists{bookmark.sty}{\usepackage{bookmark}}{\usepackage{hyperref}}
\IfFileExists{xurl.sty}{\usepackage{xurl}}{} % add URL line breaks if available
\urlstyle{same} % disable monospaced font for URLs
\hypersetup{
  pdftitle={Airbnb Pricing Project},
  pdfauthor={Divya Sharma},
  colorlinks=true,
  linkcolor={blue},
  filecolor={Maroon},
  citecolor={Blue},
  urlcolor={Blue},
  pdfcreator={LaTeX via pandoc}}

\title{Airbnb Pricing Project}
\author{Divya Sharma}
\date{}

\begin{document}
\maketitle
\ifdefined\Shaded\renewenvironment{Shaded}{\begin{tcolorbox}[frame hidden, sharp corners, enhanced, boxrule=0pt, interior hidden, breakable, borderline west={3pt}{0pt}{shadecolor}]}{\end{tcolorbox}}\fi

\hypertarget{data-analysis-assignment-01}{%
\section{Data Analysis Assignment
01}\label{data-analysis-assignment-01}}

\emph{Divya Sharma (ds655)}

\hypertarget{airbnb-pricing-in-asheville-nc}{%
\subsection{Airbnb pricing in Asheville,
NC}\label{airbnb-pricing-in-asheville-nc}}

\hypertarget{section-1-executive-report}{%
\subsection{Section 1: Executive
Report}\label{section-1-executive-report}}

Airbnb is a popular online platform that allows people to rent out their
homes to travelers. Airbnb hosts set their own prices, but they can
benefit from having accurate information about how much to charge. This
project aims to develop a linear machine learning model to predict the
price of Airbnb listings in Asheville, NC.

The model will be trained on a dataset of Airbnb listings that includes
information such as the distance to downtown, the number of bedrooms and
bathrooms, the amenities offered, and the reviews received. The model
will then be used to predict the price of new Airbnb listings based on
the same features.

This project has the potential to benefit both Airbnb hosts and guests.
Airbnb hosts can use the model to set competitive prices that are likely
to attract guests. Airbnb guests can use the model to find the best
deals on rentals.

\hypertarget{benefits-of-this-project}{%
\subsubsection{1.1 Benefits of this
Project}\label{benefits-of-this-project}}

The Airbnb price prediction project has a number of potential benefits,
including:

\begin{itemize}
\tightlist
\item
  Helping Airbnb hosts to set competitive prices and improve their
  listings. By understanding how much guests are willing to pay for
  different types of listings and amenities, Airbnb hosts can set their
  prices accordingly. This can help them to attract more guests and
  increase their earnings.
\item
  Helping Airbnb guests to find the best deals on rentals. By using the
  model to predict the price of different Airbnb listings, guests can
  find the best deals on rentals that meet their needs. This can help
  them to save money on their vacations.
\item
  Conducting research on the impact of Airbnb on the local economy and
  community. The data collected for this project can be used to conduct
  research on the impact of Airbnb on the local economy and community.
  For example, researchers could look at how Airbnb rentals are
  distributed across different neighborhoods and how Airbnb prices
  compare to hotel prices.
\end{itemize}

\hypertarget{the-model}{%
\subsubsection{1.2 The Model}\label{the-model}}

The model that was used to analyze the Airbnb dataset is a linear
regression model. Linear regression is a simple but powerful machine
learning algorithm that can be used to predict the value of a variable
(the price) based on the values of other variables (the information we
have on the listings).

Linear Regression is a simple predictive model that can easily be
interpreted

The variables that are included in the model are:

\begin{itemize}
\tightlist
\item
  \texttt{Room\ Details}: The room type (private room, shared room,
  entire home) and number of beds/bathrooms/bedrooms
\item
  \texttt{Location}: Geographic information such as the distance to
  downtown Asheville, and the locality
\item
  \texttt{Avalibility}: The availability of the listing
\item
  \texttt{Reviews}: Information based on reviews provided to the listing
\item
  \texttt{Host\ Details}: Host verification and contact availability
  details
\item
  \texttt{Amenities}: Whether the listing has basic amenities such as
  AC, Parking, Wifi, Microwave, allows pets or not etc.
\end{itemize}

For a more detailed description of each variable that has been fed into
the model, please refer to the technical report in section 2.

\hypertarget{how-the-model-accomplishes-the-goal}{%
\subsubsection{1.3 How the Model Accomplishes the
Goal}\label{how-the-model-accomplishes-the-goal}}

The linear regression model works by fitting a line to the data. The
slope of the line represents the relationship between the target
variable (price) and each of the predictor variables. For example, the
slope of the line that represents the relationship between price and
distance to downtown would be positive, because listings that are closer
to downtown are generally more expensive.

Once the model has been trained, it can be used to predict the price of
new Airbnb listings by passing the values of the predictor variables to
the model. For example, to predict the price of a listing that is 1 mile
from downtown, has 2 bedrooms, 1 bathroom, and offers wifi and parking,
the model would use the following equation:

\begin{verbatim}
price = predicted_price = 
  intercept 
  + (importance of Room Details * Room Details) 
  + (importance of Location * Location) 
  + (importance of Avalibility * Avalibility) 
  + (importance of Reviews * Reviews) 
  + (importance of Host Details * Host Details)
  + (importance of Amenities * Amenities)
\end{verbatim}

The values of the intercept and the slopes would be determined by the
model during the training process.

It is important to note that linear regression is a simple model and may
not be able to accurately predict the price of all Airbnb listings.
However, it is a good starting point for developing a predictive model.
The accuracy of the model can be improved by using more features and by
using more complex machine learning algorithms. The model will learn the
importance of each factor by training on a dataset of Airbnb listings.
Once the model is trained, it can be used to predict the price of new
Airbnb listings by passing the values of the factors to the model.

Example:

Imagine that you are an Airbnb host and you want to set a price for your
listing. You could use the linear regression model to predict how much
guests are willing to pay for your listing based on its distance to
downtown, the number of bedrooms and bathrooms, the amenities offered,
and the reviews received.

This information could help you to set a competitive price for your
listing and maximize your earnings.

\hypertarget{metrics}{%
\subsubsection{1.4 Metrics}\label{metrics}}

To justify my model, I would use the following model metrics:

\begin{itemize}
\item
  \textbf{R-squared:} R-squared can be interpreted as the percentage of
  variation in the target variable (price) that is explained by the
  predictor variables (room type, bedrooms, dist\_to\_dt, etc.).
  R-squared is a measure of how well the model fits the data, and ranges
  from 0 to 1, with a higher value indicating a better fit. In this
  case, the R-squared value is 0.5865, which means that the model
  explains \textasciitilde60\% of the variation in the price of Airbnb
  listings.
\item
  \textbf{F-statistic:} The F-statistic is a test of whether the model
  is a significant improvement over a simpler model, such as a model
  with no predictor variables. In this case, the F-statistic is 93.31
  and the p-value is less than 2.2e-16, which means that the model is a
  significant improvement over a simpler model.
\end{itemize}

Assuming that the model has been trained on a dataset of Airbnb listings
in Asheville, NC, we can predict that the price of a listing with 2
bedrooms, 2 bathrooms, and a distance to downtown of 1 mile would be
around \$200 per night. It is important to note that this is just a
prediction. The actual price of the listing may vary depending on other
factors, such as the time of year, the amenities offered, and the
reviews received.

\hypertarget{section-2-technical-report}{%
\subsection{Section 2: Technical
Report}\label{section-2-technical-report}}

\hypertarget{section-2.1-model-selection}{%
\subsubsection{Section 2.1 Model
Selection}\label{section-2.1-model-selection}}

\hypertarget{section-2.2-data-cleaning-and-eda}{%
\subsubsection{Section 2.2 Data Cleaning and
EDA}\label{section-2.2-data-cleaning-and-eda}}

The data used for this analysis is from
\href{http://insideairbnb.com/get-the-data}{Inside Airbnb},
specifically, from
\href{https://anlane611.github.io/ids702-fall23/DAA/listings.csv}{here}.
The data contains basic details about Airbnbs listed in Asheville, North
Carolina. The data dictionary for this dataset can be found
\href{https://docs.google.com/spreadsheets/d/1iWCNJcSutYqpULSQHlNyGInUvHg2BoUGoNRIGa6Szc4/edit\#gid=1322284596}{here}

There are 3,239 listings of Airbnbs which consist of mostly entire homes
(87\%), some Private Rooms (12\%) and very few hotel rooms and shared
rooms(\textasciitilde1\%).

\begin{itemize}
\tightlist
\item
  \textbf{2.1.1 Location - adding Distance to downtown (dist\_to\_dt)
  based on Latitude and Longitude data}

  \begin{itemize}
  \tightlist
  \item
    Using the \texttt{distm()} function in the \texttt{geosphere}
    library, we can calculate the distance of the latitude and longitude
    of the Airbnb to the corresponding latitude and longitude of
    Downtown, Asheville. This gives us the \texttt{dist\_to\_dt} column
    which is highly significant while calculating the price
  \end{itemize}
\item
  \textbf{2.1.2 Cleaning the Price variable}

  \begin{itemize}
  \tightlist
  \item
    The \texttt{price} variable contains the price in comma separated
    USD values, so the data has to be cleaned and made numeric
  \end{itemize}
\item
  \textbf{2.1.3 Cleaning the bathrooms count}

  \begin{itemize}
  \tightlist
  \item
    The bathrooms column is text and contains a mix of numbers (1.5) and
    text (half) values. These are converted to the corresponding numeric
    values (0.5)
  \end{itemize}
\item
  \textbf{2.1.4 Host Verifications}

  \begin{itemize}
  \tightlist
  \item
    The Host Verifications contains json type formatted lists of
    combinations of email, work email, and phone. This is split into two
    binary columns - host\_verification\_email and
    host\_verification\_phone
  \end{itemize}
\item
  \textbf{2.1.5 True/False columns}

  \begin{itemize}
  \tightlist
  \item
    Columns like has\_availability, host\_identity verified,
    is\_superhost have values `t' and `f' for true and false, and also
    contain blanks. These have been converted into 1s (for true) and 0s
    (for false and blanks)
  \end{itemize}
\item
  \textbf{2.1.6 Amenities}

  \begin{itemize}
  \tightlist
  \item
    The Amenities column has json formatted lists of amenities. We have
    taken some of the most relevant Amenities such as Wifi, Parking, Air
    Conditioning, Kitchen, Pet friendliness, Microwave, Refrigerator,
    TV, and Heating, and created columns such as has\_wifi, has parking
    etc with 1s and 0s
  \item
    \emph{The data showed that all the listings have TVs, so the entire
    column was coming as 1, so we removed that column from this
    analysis}
  \end{itemize}
\end{itemize}

\hypertarget{eda}{%
\subsubsection{EDA}\label{eda}}

The Price ranges from \$14/Night to \$2,059/Night. The average price is
\$180/Night (indicated by the red line in the chart).

\includegraphics[width=5.8125in,height=5.8125in]{IDS702_DA_Assignment_Document_files/mediabag/c849b1ef-2315-40d7-a.png}

\hypertarget{section-2.3-inputs-to-the-model}{%
\subsubsection{Section 2.3 Inputs to the
model}\label{section-2.3-inputs-to-the-model}}

\begin{itemize}
\tightlist
\item
  \texttt{Room\ type}: The type of Airbnb listing, such as private room,
  entire home, or shared room.
\item
  \texttt{Bedrooms}: The number of bedrooms in the Airbnb listing.
\item
  \texttt{Dist\_to\_dt}: The distance to downtown Asheville.
\item
  \texttt{Bathrooms\_numeric}: The number of bathrooms in the Airbnb
  listing.
\item
  \texttt{Accommodates}: The maximum number of guests that the Airbnb
  listing can accommodate.
\item
  \texttt{Beds}: The number of beds in the Airbnb listing.
\item
  \texttt{Minimum\_nights}: The minimum number of nights that guests can
  stay in the Airbnb listing.
\item
  \texttt{Has\_availability}: Whether or not the Airbnb listing is
  available on the date that the prediction is being made.
\item
  \texttt{Number\_of\_reviews}: The number of reviews that the Airbnb
  listing has received.
\item
  \texttt{Review\_scores\_rating}: The average rating of the reviews
  that the Airbnb listing has received.
\item
  \texttt{Reviews\_per\_month}: The average number of reviews that the
  Airbnb listing receives per month.
\item
  \texttt{Review\_scores\_location}: The average rating of the Airbnb
  listing's location.
\item
  \texttt{Review\_scores\_value}: The average rating of the Airbnb
  listing's value.
\item
  \texttt{Review\_scores\_communication}: The average rating of the
  Airbnb listing's communication.
\item
  \texttt{Review\_scores\_checkin}: The average rating of the Airbnb
  listing's checkin process.
\item
  \texttt{Review\_scores\_cleanliness}: The average rating of the Airbnb
  listing's cleanliness.
\item
  \texttt{Review\_scores\_accuracy}: The average rating of the Airbnb
  listing's accuracy.
\item
  \texttt{Host\_has\_profile\_pic}: Whether or not the Airbnb host has a
  profile picture.
\item
  \texttt{Host\_identity\_verified}: Whether or not the Airbnb host's
  identity has been verified.
\item
  \texttt{Host\_verification\_email}: Whether or not the Airbnb host's
  email address has been verified.
\item
  \texttt{Host\_verification\_phone}: Whether or not the Airbnb host's
  phone number has been verified.
\item
  \texttt{Host\_has\_profile\_pic}: Whether or not the Airbnb host has a
  profile picture.
\item
  \texttt{Host\_identity\_verified}: Whether or not the Airbnb host's
  identity has been verified.
\item
  \texttt{Host\_is\_superhost}: Whether or not the Airbnb host is a
  Superhost.
\item
  \texttt{Has\_ac}: Whether or not the Airbnb listing has air
  conditioning.
\item
  \texttt{Has\_parking}: Whether or not the Airbnb listing has parking.
\item
  \texttt{Has\_wifi}: Whether or not the Airbnb listing has wifi.
\item
  \texttt{Has\_kitchen}: Whether or not the Airbnb listing has a
  kitchen.
\item
  \texttt{Has\_pets}: Whether or not the Airbnb listing allows pets.
\item
  \texttt{Has\_microwave}: Whether or not the Airbnb listing has a
  microwave.
\item
  \texttt{Has\_refrigerator}: Whether or not the Airbnb listing has a
  refrigerator.
\item
  \texttt{Has\_heating}: Whether or not the Airbnb listing has heating.
\end{itemize}

\hypertarget{section-2.4-model-performance}{%
\subsubsection{Section 2.4 Model
Performance}\label{section-2.4-model-performance}}

\hypertarget{section-2.5-conclusion}{%
\subsubsection{Section 2.5 Conclusion}\label{section-2.5-conclusion}}



\end{document}
